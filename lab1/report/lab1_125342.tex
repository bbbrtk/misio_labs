\documentclass{article}

\usepackage[T1]{fontenc}
\usepackage{polski}
\usepackage[utf8]{inputenc}

\usepackage{hyperref}
\usepackage[legalpaper, margin=1in]{geometry}

\title{MISIO 2020 - Inteligentne Agenty}
\author{Bartosz Sobkowiak 125342}
\date{12.02.2020}

\usepackage{natbib}
\usepackage{graphicx}

\begin{document}

\maketitle


\section{Pytania}
\begin{enumerate}
    \item \textbf{Jakie cechy ma to środowisko?} \\
    Całkowicie obserwowalne, deterministyczne, epizodyczne, statyczne, dyskretne, jednoagentowe.
    
    \item \textbf{Jeden z agentów okazał się dużo lepszy. Dlaczego?} \\
    Drugi agent przestaje przemieszczać się lewo-prawo, jeśli oba pola są czyste. Wykonuje więc mniej ruchów (dostaje mniej punktów ujemnych), a osiąga taki sam efekt jak pierwszy agent, który cały czas sprawdza każda pole.
    
    \item \textbf{Czy agenty ReflexVacuumAgent lub ModelBasedVacuumAgent są racjonalne? Uzasadnij.} \\ 
    Agent pierwszy pomimo posiadania wiedzy dotyczącej stanu pól i tak się przemieszcza, więc nie maksymalizuje w ten sposób wartości miary jakości. Drugi agent stara się maksymalizować poprzez wykonywanie mniejszej liczby niepotrzebnych ruchów, lecz również nie jest racjonalny.
    
    \item \textbf{Czy dla tego środowiska istnieje racjonalny agent odruchowy? Uzasadnij.} \\
    Jeśli środowisko jest calkowicie obserwowalne, to agent odruchowy może być racjonalny (wykład, slajd 32). To środowisko jest obserwowalne, więc taki agent istnieje.
    
    \item \textbf{Jakie ma ono cechy?} \\
	Całkowicie obserwowalne, stochastyczne, epizodyczne, statyczne, dyskretne, jednoagentowe.

\end{enumerate}

\section{Kod agenta}
to Kod

\section{Histogram}
to histiogram

\end{document}
